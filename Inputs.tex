\documentclass{article}

\usepackage[latin1]{inputenc}
\usepackage[portuges,brazilian]{babel}

\usepackage{enumitem, kantlipsum}

\setlist{  
	listparindent=\parindent, 
	parsep=0pt, 
	labelsep=2em, 
	itemindent=\parindent }

\author{Lucas Gon�alves Serrano - RA: 12.01328-5 \and Fl�via Janine B�o Rosante - RA: 13.03188-0 \and Erica Yumi Kido - RA: 13.02422-0}

\begin{document}
\date{\today}
\title{PIO\\Entradas e Driver}

\maketitle

	\section{Input}
	\textbf{1 - Entrada Digital}
	
	O sinal el�trico chamado de "on-off" que ao se medir um sensor s� existe dois tipos de informa��es, por exemplo, 0V ou 12V, ou seja, ligado ou desligado.
	
	\subsection{Pull-UP / Pull-Down}
	\textbf{2 - Valores resistores} 
	
	O Pull-Up tem uma resist�ncia m�nima de 70 KOhms e m�xima de 130 KOhms. Assim como o Pull-Down
	
	\subsection{Deboucing + Glitch}
	\textbf{3 - Divisor de Clock}\\ 
	Este registrador aceita um valor de 2\textsuperscript{14}. \ \\
	\textbf{4 - Interpreta��o carta de tempo}\\ 
	O Debouncing e o Glitch s�o utilizados para desconsiderar qualquer oscila��o na onda de input do bot�o que n�o represente uma pressionada real do bot�o.
	\subsection{Programa��o}
	\textbf{5 - While(1)}\\
	Uma possibilidade para n�o necessitar a checagem do status do bot�o repetidamente seria utilizar uma fun��o chamada quando o bot�o � pressionado, mudando uma vari�vel do sistema que essa em si seria checada.
	\section{Driver} 
	\subsection{PMC}
	\textbf{6 - Include} \\
	O include com "" � utilizado para bibliotecas geralmente na pasta do projeto em si, s�o normalmente bibliotecas feitas pelo usu�rio, j� com < > � utilizados para bibliotecas do sistema, o pr�-compilador as procura em pastas pr� determinadas, onde est�o as bibliotecas mais comumente usadas (como em nossos projetos a biblioteca \textbf{asf.h}).
\end{document}