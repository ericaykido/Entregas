\documentclass{article}

\usepackage[latin1]{inputenc}
\usepackage[portugues,brazilian]{babel}


\usepackage{enumitem, kantlipsum}

\setlist{  
	listparindent=\parindent, 
	parsep=0pt, 
	labelsep=2em, 
	itemindent=\parindent }


\begin{document}
	\title{PIO\\
		Entradas e Driver}
\author{Lucas Gonçalves Serrano - RA: 12.01328-5 \and Flávia Janine Béo Rosante - RA: 13.03188-0 \and Erica Yumi Kido - RA: 13.02422-0}
	\maketitle

	\section{Input}
	\textbf{1 - Entrada Digital}
	
	O sinal elétrico chamado de "on-off" que ao se medir um sensor só existe dois tipos de informações, por exemplo, 0V ou 12V, ou seja, ligado ou desligado.
	
	\subsection{Pull-UP / Pull-Down}
	\textbf{2 - Valores resistores} 
	
	O Pull-Up tem uma resistência mínima de 70 KOhms e máxima de 130 KOhms. Assim como o Pull-Down
	
	\subsection{Deboucing + Glitch}
	\textbf{3 - Divisor de Clock}\\
	\textbf{4 - Interpretação carta de tempo}
	
	\subsection{Programação}
	\textbf{5 - While(1)}
	
	\section{Driver}
	\subsection{PMC}
	\textbf{6 - Include}
\end{document}